\chapter{Zaključak i budući rad}
		
		Naša grupa je kao zadatak dobila razvoj web aplikacije za Kamp mlade nade. Ta aplikacija trebala bi služiti organizatorima kampa, sudionicima i animatorima kampa, u svrhe bolje organizacije. Dvije ključne stavke bile su raspodjela sudionika i animatora po grupama te dodavanje aktivnosti u raspored koji je uvijek na raspolaganju svim korisnicima sustava. Dodatno, trebali smo omogućiti davanje povratnih ocjena iskustva, kako bi organizatori dobili povratne informacije o uspješnosti kampa. Nakon 17 tjedana uspjeli smo razviti traženu aplikaciju. Sam razvoj projekta odvijao se kroz 2 faze. 
        
        U prvoj fazi obavili smo međusobno upoznavanje članova tima, podjelu zadataka te smo krenuli s proučavanjem zadatka. Usporedno s proučavanjem zadatka krenuli smo i s izradom dokumentacije koja nam je kasnije dobro poslužila i u rješavanju implementacijskih problema. Iako smo se prvi put susreli sa obrascima uporabe, sekvencijskim dijagramima i dijagramima razreda, ubrzo smo shvatili njihovu važnost u daljnjem procesu razvoja. Nadalje, međusobno smo se podijelili u 3 tima: backend, frontend i dokumentacija, jer smo shvatili da će veliki naglasak biti upravo na izradi dokumentacije, pogotovo u prvoj fazi.
        
        Nakon inicijalne podjele rada i tima te rješavanja, krenuli smo
		u drugu faza projekta. Iako nešto kraća od prve, bila je puno intezivnija po pitanju rada svih članova. Neki članovi koji su imali više prijašnjeg iskustva u tehnologijama koje smo koristili pomogli su manje iskusnim članovima pa nismo morali trošiti previše vremena na učenje novih tehnologija. Osim realizacije rješenja, u drugoj fazi je bilo potrebno dokumentirati ostale UML dijagrame i izraditi popratnu dokumentaciju kako bi budući korisnici mogli lakše koristiti ili vršiti preinake na sustavu. Dobro izrađen kostur projekta uštedio nam je mnogo vremena prilikom izrade aplikacije te smo izbjegli moguće pogreške u izradi koje bi bile vremenski skupe za ispravljanje u daljnjoj fazi projekta.
		
		Komunikacija među članovima grupe je bila putem Slacka gdje je svaka podgrupa imala svoji kanal u kojima je mogla raspravljati detalje vezane uz svoj dio impelemntacije. Sastanke smo odrađivali svaki tjedan, prvo uživo, a kasnije preko servisa MS Teams. Svaki član obavještavao je o svojem napretku, a dodatno smo se i dogovarali koji su nam zadaci za sljedeći tjedan.
		
		Moguće proširenje postojeće inačice sustava bila bi izrada mobilne aplikacije čime bi aplikacija bila dostupna većem broju korisnika i daljnji rad na UI/UX dizajnu čime bi korištenje aplikacije bilo još intuitivnije.
		
		Sudjelovanje na ovakvom projektu bilo je vrijedno iskustvo svim članovima tima jer smo kroz intenzivnih nekoliko tjedana rada iskusili zajednički rad na istom projektu. Također, osjetili smo važnost dobre vremenske organiziranosti i koordiniranosti između članova tima. Zadovoljni smo postignutim bez obzira na moguć prostor za usavršavanje aplikacije što je posljedica neiskustva na takvim i sličnim projektima.
		
		Sve su funkcionalnosti implementirane u ostvarenoj aplikaciji.

		
		\eject 
